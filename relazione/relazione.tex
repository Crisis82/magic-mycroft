\documentclass[12pt,a4paper]{article}

\usepackage[utf8]{inputenc}
\usepackage[T1]{fontenc}

\usepackage{fancyhdr}
\pagestyle{fancy}
\setlength{\headheight}{16pt}

\usepackage{graphicx}
\graphicspath{ {./img/} }

\usepackage[backend=bibtex,style=ieee]{biblatex}
\addbibresource{relazione.bib}

\title{Sviluppo di uno Smart Mirror ed integrazione di un AI}
\author{Filippo Merlo}
\date{2021-2022}

\begin{document}

\maketitle
\newpage
\tableofcontents
\newpage

\section{Introduzione}

Lo sviluppo di prodotti per il mercato dell'\textit{Internet of Things} \`e in
continua crescita ed oggetti che fino a non molto tempo fa erano completamente
inanimati, ora sono dotati di sensori, memorie e processori per l'implementazione
di funzionalit\'a aggiuntive. Da qui \`e nato appunto il termine \textit{Smart},
necessario a descrivere un oggetto che ora possiede propriet\`a `intellettive`.
L'integrazione tra tecnologia e quotidianet\`a \`e stato il perfetto mix per
un rapido sviluppo del mondo dell'IoT, infatti la proprosta di oggetti Smart che
rendessero semplici ed automatici i ripetitivi compiti di ogni giorno \`e stata
ben accetta dalle persone, permettendo alla tecnologia di diventare parte integrante
e fondamentale delle nostre vite.

\subsection{Obiettivo}

In una prospettiva mondiale, nella mole di proposte tecnologiche, vi sono ancora
alcuni oggetti che resistono al `trend'dell'IoT e tra questi possiamo trovare lo
specchio. Utilizzato molteplici volte durante il giorno da ogni tipo di persona e
presente in varie tipologie di forme e dimensioni, \'e ormai un `must have'nella vita
odierna e ormai viene utilizzato senza quasi rendercene conto, come fosse una nostra
fisica estensione visiva.

Una buona parte del nostro tempo viene spesa davanti ad esso, per cui l'obiettivo
di questo progetto \'e stato quello estendere le informazioni visive fornite da uno
specchio, attraverso la realizzazione di una sua versione Smart. Il prototipo sviluppato
\`e stato creato concentrandosi principalmente su un suo possibile utilizzo, nello
specifico quello all'interno delle proprie abitazioni, durante le varie ruotine
giornaliere. Normalmente \'e presente almeno uno specchio per bagno, senza contare
altri possibili nelle camere e guardaroba e viene utilizzato per assisterci in compiti
come sistemarci il volto o i capelli, provare l'abbigliamento e controllare di avere un
aspetto in ordine. Si guarda nello specchio anche mentre si svolgono altri compiti, in cui
lo specchio non risulta fondamentale ma \'e sempre presente data la sua posizione, come
ad esempio lavarsi le mani o i denti ed asciugarsi i capelli. La cosa che accomuna tutte
queste azioni \'e il ruolo dello specchio, che risulta essere sempre presente ma come 
assistente per azioni primarie, in cui abbiamo spesso le mani occupate. Dati tutti questi
fattori risulta immediato pensare a come uno smart mirror che fornisce informazioni aggiuntive
come l'ora, la presenza di notifiche mail o messaggi non lette, prossimi eventi del calendario,
previsioni meteo, per arrivare anche alla presentazioni di notizie, si integri quasi perfettamente
con un assistente digitale, controllabile con la voce, rendendo cos\`i possibile gestire
ulteriori task comporaneamente al compito che si sta svolgendo.

\section{Analisi}

\subsection{Requisiti}

Per il corretto sviluppo del progetto, oltre ad un normalissimo specchio, risultano necessari un computer
ed eventuali periferiche e sensori, legati ai moduli da visualizzare sullo specchio, dove per \textit{modulo}
si intende un servizio che viene eseguito in background dal computer e presenta sullo schermo le informazioni
raccolte. Il formato e la quantit\`a (dettaglio) di informazioni presentate sono personalizzati in base alla
scelta dell'utente come lo \`e la scelta dei moduli.

Per il corretto funzionamento della maggior parte dei moduli risulta ovviamente necessario essere dotati di una
connessione internet.

\subsection{Architettura}

L'archittetura del progetto pu\'o essere distinta tra componenti fisiche e componenti virtuali, per cui:

\begin{itemize}
  \item \textbf{fisiche}: specchio, raspberry pi, modulo wifi/cavo ethernet
  \item \textbf{virtuali}: MagicMirror, moduli, assistente digitale
\end{itemize}

\subsubsection{Smart Mirror}

Sul mercato sono gi\`a presenti prototipi pronti all'acquisto e funzionanti, tuttavia se si vuole realizzazione
una soluzione \textit{home made} il processo risulta semplice e con un minimo di ricerca su Internet a riguardo,
\`e possibile trovare molteplici guide, pi\`u o meno dettagliate, che forniscono soluzioni \textit{plug and play}.

Per il seguente progetto si \`e scelto come base di sviluppo MagicMirror$^2$\cite{MagicMirrorRepo}, un piattaforma
\textit{open-source} modulare per la gestione e personalizzazione completa del proprio Smart Mirror. Il core di 
MagicMirror$^2$ contiene una forte API che permette a sviluppatori di terze parti di implementare i propri moduli
da integrare, infatti sul sito ufficiale\cite{MagicMirrorSite} \`e presente una sezione con la lista di tutti i
moduli sviluppati da terzi e pubblicati\cite{MagicMirrorModules}. In questo progetto i moduli utilizzati sono i seguenti:

\begin{itemize}
  \item orologio
  \item calendario
  \item meteo
  \item news
  \item pagine
  \item indicatore di pagina
  \item compleanni
  \item controllo remoto
\end{itemize}



\subsubsection{Intelligenza Artificiale}


\section{Implementazione}
\subsection{MagicMirror$^2$}
\subsection{Moduli}
\subsection{Mycroft}

\section{Conclusioni}
\subsection{Considerazioni}
\subsection{Limiti e miglioramenti}

\printbibliography

\end{document}
